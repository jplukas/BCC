\hypertarget{sequencias}{%
\section{Sequencias}\label{sequencias}}

Uma sequência é uma função de inteiros positivos (Naturais) a valores
reais. Se a cada número natural \(n\) é associado um número \(a_n\),
diz-se que os valores \(a_n\) formam uma sequência infinita. Ordenam-se
os números segundo seus índices: \(a_1, a_2, \dots, a_n, \dots\)\\
Notação para sequência:
\(\{a_1, a_2, \dots, a_n \dots\} = \{a_n\}_{n \in \mathbb{N}} = \{a_n\}_{n=1}^\infty = \{a_n\}\)\\
Todos esses símbolos denotam a mesma sequência, e o termo \(a_n\) é
chamado de o n-ésimo termo da sequência.

Exemplos: 1. Se \(a_n = 2^n\), então a sequência é
\(\{2, 4, 8, \dots\}\) 2. Se \(a_n = n\), então a sequência é
\(\{1, 2, 3, \dots\}\) 3. Dada a sequência
\(\{0, \frac{1}{2}, \frac{2}{3}, \frac{3}{4}, \frac{4}{5}, \dots\}\),
qual seria o n-ésimo termo? \(a_n = \frac{n-1}{n}\), com \(n \geq 1\).
4. Dada a sequência
\(\{\frac{1}{2}, -\frac{1}{2}, \frac{1}{4}, -\frac{3}{4}, \frac{1}{8}, -\frac{7}{8}, \dots\}\),
qual seria o n-ésimo termo? \(a_{2n - 1} = \frac{1}{2^n}\), e
\(a_{2n} = -\frac{2n + 1}{2^n}\), para \(n \geq 1\). 5. Outra forma de
se escrever uma sequência é por meio de uma relação de recorrência:
\(a_1 = 2\), \(a_{n+1} = \sqrt{2 + a_n}\). Assim, a sequência é:
\(\{\sqrt{2}, \sqrt{2 + \sqrt{2}}, \sqrt{2 + \sqrt{2 + \sqrt{2}}}\}\).\\
Um outro exemplo importante é a sequência de Fibonacci dada por:
\(a_1 = 1\), \(a_2 = 1\), e \(a_n = a_{n - 1} + a_{n - 2}, \ n \geq 3\).
Assim, a sequência de Fibonacci é
\(\{1, 1, 2, 3, 5, 8, 13, 21, 34, \dots\}\).

Definição: A sequência \(\{a_n\}\) tem limite \(L\) finito, e escrevemos
\(\lim_{n \to \infty} a_n = L\), se \(a_n\) se aproxima de \(L\) ao
passo quando \(n\) fica maior. Mais formalmente, \[
\lim_{n \to \infty} a_n = L
\iff
\forall \epsilon(\epsilon > 0 \implies \exists n_0(\forall n(n \geq n_0 \implies |L - a_n| \leq \epsilon)))
\] Se o limite \(L\) existe, dizemos que a sequência é convergente (mais
ainda, dizemos que \(a_n\) é convergente a \(L\)). Caso contrário,
dizemos que a sequência é divergente.

Se uma sequência \(a_n\) provém de uma função
\(f:[0, +\infty[ \to \mathbb{R}\), isto é, \(a_n = f(n)\) e é verdade
que \(\lim_{x \to + \infty} f(x) = L\), então \(a_n\) também converge a
\(L\). Portanto, podemos aplicar a teoria sobre limites de funções
quando falamos de sequências também. Prova: TODO

Exemplos:

\begin{enumerate}
\def\labelenumi{\arabic{enumi}.}
\setcounter{enumi}{6}
\item
  \(\lim_{n \to + \infty} e^{-n} = 0\)
\item
  \[
  \lim_{n \to + \infty} \dfrac{\ln n}{n^a} \overset{\text{L'hopital}}{=} \lim_{n \to + \infty} \dfrac{1}{n \cdot an^{a-1}} = \lim_{n \to + \infty} \dfrac{1}{an^a} = 0
  \]
\item
  \[
  \lim_{n \to + \infty} \sqrt[n]{a} = \lim_{n \to + \infty} a ^{\frac{1}{n}} \overset{\text{cont. da exp.}}{=} a ^ {\lim_{n \to + \infty} \frac{1}{n}} = a^0 = 1 \ (\text{para } a > 0) 
  \]
\item
  \[
  \lim_{n \to \infty} \left(\dfrac{n-1}{n}\right)^n = \ ?
  \] Seja \(f(x) = \left(\dfrac{x-1}{x}\right)^x\). Entao vamos calcular
  o limite \(\lim_{x \to \infty} f(x)\). \[
  \ln\left(\left(\dfrac{x-1}{x}\right)^x\right) = 
  x(\ln(x-1) - \ln(x))\\
  x(\ln(x-1) - \ln(x)) = 
  \dfrac{\ln(x-1) - \ln(x)}{\dfrac{1}{x}}
  \] Entao, temos \[
  \lim_{x \to \infty} \dfrac{\ln(x-1) - \ln(x)}{\dfrac{1}{x}}
  \overset{\text{L'Hopital}}{=}
  \lim_{x \to \infty}
  \dfrac{\dfrac{1}{x - 1} - \dfrac{1}{x}}{-x^{-2}}\\
  \lim_{x \to \infty}
  \dfrac{\dfrac{1}{x - 1} - \dfrac{1}{x}}{-x^{-2}} =
  \lim_{x \to \infty}
  x - \dfrac{x^2}{x-1}\\
  \lim_{x \to \infty}
  x - \dfrac{x^2}{x-1} = 
  \lim_{x \to \infty}
  \dfrac{x^2 - x - x^2}{x-1}\\
  \lim_{x \to \infty}
  \dfrac{x^2 - x - x^2}{x-1} = 
  \lim_{x \to \infty}
  \dfrac{- x}{x-1} = -1
  \] Entao, como \(\lim_{x \to \infty} \ln (f(x)) = -1\), temos que
  \(\lim_{x \to \infty} f(x) = e^{-1}\).
\item
  Calcule \(\lim_{n \to \infty} \left(\dfrac{n+a}{n}\right)^n\). \[
  \ln \left(\dfrac{n+a}{n}\right)^n = 
  \dfrac{(\ln (n + a) - \ln n)}{n^{-1}}
  \] \[
  \lim_{x \to \infty}
  \dfrac{(\ln (x + a) - \ln x)}{x^{-1}}
  \overset{\text{L'Hopital}}{=}
  \lim_{x \to \infty}
  \dfrac{(x+a)^{-1} - x^{-1}}{-x^{-2}} \\
  \lim_{x \to \infty}
  \dfrac{(x+a)^{-1} - x^{-1}}{-x^{-2}} = 
  \lim_{x \to \infty}
  x - \dfrac{x^2}{x+a}\\
  \lim_{x \to \infty}
  \dfrac{x^2 + ax - x^2}{x+a}\\
  \lim_{x \to \infty}
  \dfrac{x^2 + ax - x^2}{x+a} = 
  \lim_{x \to \infty}
  \dfrac{ax}{x+a} \\
  \lim_{x \to \infty}
  \dfrac{ax}{x+a} = 
  a \lim_{x \to \infty}
  \dfrac{x}{x+a} = a \cdot 1 = a
  \] Como \(\lim_{x \to \infty} \ln (f(x)) = a\), temos que
  \(\lim_{x \to \infty} f(x) = e^a\).
\end{enumerate}

\hypertarget{sequuxeancias-monotuxf4nicas}{%
\subsection{Sequências monotônicas}\label{sequuxeancias-monotuxf4nicas}}

Definição: uma sequência \(\{a_n\}\) é dita crescente se (e somente se)
\(a_n < a_{n + 1}\), para todo \(n \geq 1\).\\
Uma sequência \(\{a_n\}\) é dita decrescente se (e somente se)
\(a_n > a_{n + 1}\), para todo \(n \geq 1\).\\
Uma sequência é dita monotônica se for crescente ou decrescente.

Exemplo:

\begin{enumerate}
\def\labelenumi{\arabic{enumi}.}
\setcounter{enumi}{11}
\tightlist
\item
  Seja \(f:[1, \infty) \to \mathbb{R}\) uma função crescente
  \(f^\prime(x) > 0\), então \(a_n = f(n)\) será uma sequência
  crescente. A situação é a análoga para funções decrescentes.\\
  Prova: Pelo TVM, temos \[
  f(n + 1) = f^\prime(c) \cdot (n + 1 - n) + f(n)\\
  f(n + 1) = f^\prime(c) + f(n)
  \] Onde \(n < c < n + 1\). Porém, como \(f^\prime(x) > 0\) para todo
  \(x\) no domínio, em especial \(x = c\), temos que
  \(f^\prime(c) > 0\), e então \[
  f(n + 1) = f^\prime(c) + f(n) > 0 + f(n) \\
  f(n + 1) > f(n)
  \]
\item
  Mostre que a sequência \(\left\{ \dfrac{e^{-n}}{n} \right\}\) é
  decrescente: \[
  f(x) = \dfrac{e^{-x}}{x}\\
  f^\prime(x) = \dfrac{-xe^{-x} -e^{-x}}{x^2}\\
  f^\prime(x) = -e^{-x} \dfrac{(x + 1)}{x^2}\\
  f^\prime(x) \leq 0 \iff 
  \dfrac{(x + 1)}{x^2} \geq 0\\
  f^\prime(x) \leq 0 \iff 
  x \geq -1
  \] Portanto, para \(x \geq 0\), a função é decrescente. Logo, a
  sequência também é.
\end{enumerate}

\hypertarget{sequuxeancias-limitadas}{%
\subsection{Sequências limitadas}\label{sequuxeancias-limitadas}}

Uma sequência é dita limitada superiormente se existe
\(M \in \mathbb{R}\) tal que \(\forall n(a_n \leq M)\). Analogamente,
uma sequência é dita limitada inferiormente se existe
\(m \in \mathbb{R}\) tal que \(\forall n(a_n \geq m)\).

Teorema das sequências monotônicas: Toda sequência crescente e limitada
superiormente é convergente.\\
Prova: Seja \(a_n\) uma sequência crescente e limitada superiormente
(mais especificamente, por \(M \in \mathbb{R}\)). Pela completude dos
números reais, como o conjunto \(\{a_n\}\) é limitado superiormente,
deve ter um supremo, isto é, um número \(L \in \mathbb{R}\) tal que
\(\forall x(x \in \{a_n\} \implies x \leq L)\). É fácil ver que no nosso
caso, a nossa sequência \(a_n\) converge a \(L\). Vamos supor então que
não seja esse o caso, isto é: \[
\not \forall \epsilon(
    \epsilon > 0 \implies
    \exists n_0(
        \forall n(
            n \geq n_0 \implies
            |L - a_n| < \epsilon
        )
    )
)\\
\exists \epsilon(
    \epsilon > 0 \wedge
    \not \exists n_0(
        \forall n(
            n \geq n_0 \implies
            |L - a_n| < \epsilon
        )
    )
)\\
\exists \epsilon(
    \epsilon > 0 \wedge
    \forall n_0(
        \not \forall n(
            n \geq n_0 \implies
            |L - a_n| < \epsilon
        )
    )
)\\
\exists \epsilon(
    \epsilon > 0 \wedge
    \forall n_0(
        \exists n(
            n \geq n_0 \wedge
            |L - a_n| \geq \epsilon
        )
    )
)\\
\exists \epsilon(
    \epsilon > 0 \wedge
    \forall n_0(
        \exists n(
            n \geq n_0 \wedge
            L - a_n \geq \epsilon
        )
    )
)\\
\exists \epsilon(
    \epsilon > 0 \wedge
    \forall n_0(
        \exists n(
            n \geq n_0 \wedge
            a_n \leq L - \epsilon
        )
    )
)\\
\] Como \(a_n\) é crescente, para qualquer \(i < n\), temos que
\(a_i < a_n\). Em especial, escolhendo \(i \leq n_0\), temos que \[
\exists \epsilon(
    \epsilon > 0 \wedge
    \forall i(
        a_i \leq L - \epsilon
    )
)\\
\] O que contradiz a nossa afirmação que \(L\) é o supremo de \(a_n\).
Portanto, nossa hipótese que \(a_n\) não converge a \(L\) está errada, e
\(\lim_{n \to \infty} a_n = L\).

Exemplo

\begin{enumerate}
\def\labelenumi{\arabic{enumi}.}
\setcounter{enumi}{21}
\item
  Determine se a sequência \(\{a_n\}\) é convergente, onde \(a_i = 2\) e
  \(a_{n+1} = \frac{1}{2} (a_n + 6), \ \text{para } n \geq 1\).\\
  Vamos provar que a sequência é crescente e limitada superiormente.
  Primeiramente, temos que, para qualquer \(n \geq 1\), \[
  a_{n+1} - a_n = \frac{1}{2} (a_n + 6) - a_n\\
  a_{n+1} - a_n = \frac{a_n}{2} + 3\\
  a_{n+1} - a_n > 0 \iff
  \frac{a_n}{2} + 3 > 0\\
  a_{n+1} - a_n > 0 \iff
  \frac{a_n}{2} > -3\\
  a_{n+1} - a_n > 0 \iff
  a_n > -6\\
  \] Isto é, \(\{a_n\}\) é crescente se e somente se \(a_n > -6\) para
  todo \(n \in \mathbb{N}\). A seguir provamos isso por indução.\\
  Suponha que \(a_n > -6\). Então: \[
  a_{n + 1} = \frac{1}{2} (a_n + 6) >
  \dfrac{1}{2} (-6 + 6)\\
  a_{n + 1} > \dfrac{1}{2} \cdot 0\\
  a_{n + 1} > 0\\
  a_{n + 1} > -6\\
  \] Então temos que \(a_n > -6 \implies a_{n+1} > -6\). Para completar
  a indução, basta observar que \(a_1 = 2 > -6\). Provamos então que
  \(\{a_n\}\) é crescente.\\
  Vamos provar agora então que \(a_n\) é limitada superiormente, vamos
  supor, por \(6\): \[
  a_n < 6 \implies
  a_{n+1} = \frac{1}{2} (a_n + 6) <
  \dfrac{1}{2} (6 + 6)\\
  a_n < 6 \implies
  a_{n+1} <
  \dfrac{1}{2} \cdot 12\\
  a_n < 6 \implies
  a_{n+1} < 6\\
  \] Para completar nossa prova por indução, basta observar que
  \(a_1 = 2 < 6\).\\
  Provamos então que a sequência \(\{a_n\}\) é crescente e limitada.
  Portanto, pelo teorema das sequências monotônicas, ela também é
  convergente. Vamos agora descobrir para qual valor ela converge.\\
  Podemos dizer que \(\lim_{n \to \infty} a_n = L\). Porém, também é
  verdade que \(\lim_{n \to \infty} a_{n + 1} = L\), então, temos: \[
  L = \lim_{n \to \infty} a_{n + 1} = 
  \lim_{n \to \infty} \dfrac{1}{2}(a_n + 6)\\ 
  L =  
  3 + \dfrac{1}{2} \lim_{n \to \infty} a_n\\
  L =  
  3 + \dfrac{1}{2} L\\  
  3 = \dfrac{1}{2} L\\
  L = 6 
  \]
\item
  \$ \lim\_\{n \to \infty\} \dfrac{\sin(\frac{1}{2^n})}
  \{\frac{1}{2^n}\} \$

  Como \(\lim_{n \to \infty} \frac{1}{2^n} = 0\), então \[
  \lim_{n \to \infty}
  \dfrac{\sin(\frac{1}{2^n})}
  {\frac{1}{2^n}} = 
  \lim_{x \to 0} \dfrac{\sin{x}}{x}\\
  \lim_{n \to \infty}
  \dfrac{\sin(\frac{1}{2^n})}
  {\frac{1}{2^n}} = 1
  \]
\item
  \$ \lim\_\{n \to \infty\} \arctan{\left(\dfrac{1}{n}\right)} \$ Como
  \(\lim_{n \to \infty} \frac{1}{n} = 0\), então \[
  \lim_{n \to \infty}
  \arctan{
  \left(\dfrac{1}{n}\right)
  }
  = 
  \lim_{x \to 0} \arctan{(x)}\\
  \lim_{n \to \infty}
  \arctan{
  \left(\dfrac{1}{n}\right)
  }
  =0
  \]
\item
  \$

  \lim\_\{n \to \infty\} \ln{n} \$ Vamos supor que o limite exista, e
  \(\lim_{n \to \infty}\ln{n} = L\). Isso significaria que \[
  \forall \epsilon(\epsilon > 0 \implies \exists n_0(\forall n(n \geq n_0 \implies |L - \ln n| < \epsilon)))
  \]
\end{enumerate}

Porém, temos que \(\ln n\) é crescente, e então
\(n > e^L \implies \ln n > L\). Então a sequência \(\ln n\) não
converge.
