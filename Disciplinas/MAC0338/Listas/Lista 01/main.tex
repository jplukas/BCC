\documentclass{article}

% Language setting
% Replace `english' with e.g. `spanish' to change the document language
\usepackage[brazil]{babel}

% Set page size and margins
% Replace `letterpaper' with `a4paper' for UK/EU standard size
\usepackage[letterpaper,top=2cm,bottom=2cm,left=3cm,right=3cm,marginparwidth=1.75cm]{geometry}

% Useful packages
\usepackage{amsmath}
\usepackage{amsfonts}
\usepackage{graphicx}
\usepackage[colorlinks=true, allcolors=blue]{hyperref}
\usepackage{enumitem}
\usepackage{bm}

\title{
    Lista 01 - MAC0338\\
    Análise de Algoritmos
}
\author{
    João Pedro Lukasavicus Silva\\
    \href{mailto:joao.lukasavicus.silva@usp.br}
    {joao.lukasavicus.silva@usp.br}\\
    9276940
}

\begin{document}
\maketitle

\begin{enumerate}[itemsep=3ex, label=\textbf{\arabic*}.]
    \item \label{item1}
    Lembre-se que $\lg n$ denota o logaritmo na base $2$ de $n$. Usando a definicao de notacao $\mathcal{O}$, prove que:
    \begin{enumerate}
        \item
        $3^n$ nao eh $\mathcal{O}(2^n)$
      
        \textit{\textbf{Resposta:}}\\
        Vamos supor que $3^n \in \mathcal{O}(2^n)$. Isso significa que existem constantes positivas $c$ e $n_0$ tais que, para todo $n \geq n_0 \implies c 2^n \geq 3^n$. 
		Temos que:
		$$
		c 2^n \geq 3^n \iff \lg c + n \geq n \lg 3
	    $$
		$$
		c 2^n \geq 3^n \iff \lg c \geq n (\lg 3 - 1)
	    $$
		$$
		c 2^n \geq 3^n \iff \dfrac{\lg c}{\lg 3 - 1} \geq n
	    $$
		Então, dados $c$ e $n_0$ basta escolhermos $n > n_0$ tal que $n > \dfrac{\lg c}{\lg 3 - 1}$ que obtemos um contra-exemplo para a nossa suposição.
		
	    \item \label{item}
        $\log_{10}n$ eh $\mathcal{O}(\lg n)$
      
        \textit{\textbf{Resposta:}}\\
        Temos que:
        $$
        \lg n = \lg 10 \log_{10} n
        $$
        Assim, escolhendo $c = (\lg 10)^{-1}$ e $n_0 = 1$, eh facil ver que $\forall n \geq n_0 \implies \log_{10} n \leq c \lg n$ (tambem eh verdade que $\forall n \geq n_0 \implies \log_{10} n \geq c \lg n$). Portanto, pela definicao, $\log_{10}n \in \mathcal{O}(\lg n)$
        
        \item
        $\log_{10}n$ eh $\mathcal{O}(\lg n)$
      
        \textit{\textbf{Resposta:}}\\
        Ver item \ref{item}
        
    \end{enumerate}
    
    \item
    Usando a definicao de notacao $\mathcal{O}$, prove que:
    
    \begin{enumerate}
        \item
        $n^2 + 10n + 20 = \mathcal{O}(n^2)$
        
        \textit{\textbf{Resposta:}}\\
        Eh facil ver que $n^2 + 10n + 20 \leq 2n^2 \iff n^2 + 10n + 20 - 2n^2 \leq 0$. 
        Seja entao o polinomio $-n^2 + 10n + 20 = n^2 + 10n + 20 - 2n^2$. Temos que as raizes desse polinomio sao $5 \pm \sqrt{3}$. Assim, escolhendo $c = 2$ e $n_0 = 5 + \sqrt{3}$, temos que
        $$
        \forall n (n \geq n_0 \implies n^2 + 10n + 20 \leq c n^2)
        $$
        
        \item
        $\lceil n/3 \rceil = \mathcal{O}(n)$
        
        \textit{\textbf{Resposta:}}\\
        Pela definicao da operacao teto, temos que $n/3 \leq \lceil n/3 \rceil < n/3 + 1$. Portanto, $n \geq n/3 + 1 \implies n > \lceil n/3 \rceil$.\\
        Seja $1 - 2/3 n = n/3 + 1 - n$. Eh facil ver que $n/3 + 1 \leq n \iff 1 - 2/3 n \leq 0 \iff n \geq 3/2$.\\
        Assim, escolhendo $c = 1$ e $n_0 = 3$, temos que
        $$
        \forall n (n \geq n_0 \implies \lceil n/3 \rceil \leq c n)
        $$
        
        \item
        $\lg n = \mathcal{O}(\log_{10} n)$
        
        \textit{\textbf{Resposta:}}\\
        Ver item \ref{item}
        
        \item
        $n = \mathcal{O}(2^n)$
        
        \textit{\textbf{Resposta:}}\\
        Temos que $n \leq 2^n \iff n - 2^n \leq 0$. Como a primeira derivada de $n - 2^n$ eh igual a $1 - \ln 2 \cdot 2^n$ e $2^n$ eh uma funcao estritamente crescente, temos que $1 - \ln 2 \cdot 2^n$ eh estritamente decrescente, e $n \leq 1 \implies 1 - \ln 2 \cdot 2^n < 0$. Assim, podemos afirmar que $n - 2^n$ tambem eh estritamente decrescente no intervalo $[1, \infty)$.\\
        Para $n = 1$, temos que $n - 2^n = -1 < 0$. Entao, escolhendo $n_0 = 1$ e $c = 1$ podemos afirmar:
        $$
        \forall n (n \geq n_0 \implies n \leq c 2^n) 
        $$
        
        \item
        $n/1000$ nao eh $\mathcal{O}(1)$
        
        \textit{\textbf{Resposta:}}\\
        Vamos supor que $n/1000 \in \mathcal{O}(1)$. Assim, existem constantes positivas $c$ e $n_0$ tais que $\forall n (n \geq n_0 \implies n/1000 \leq c)$.
        Porem, escolhendo $n$ tal que $n \geq n_0$ e $n > 1000c$, temos que $n \geq n_0$ e $n/1000 > c$, o que contradiz a nossa suposicao.
        
        \item
        $n^2/2$ nao eh $\mathcal{O}(n)$
        
        \textit{\textbf{Resposta:}}\\
        Vamos supor que $n^2/2 \in \mathcal{O}(n)$. Assim, existem constantes positivas $c$ e $n_0$ tais que $\forall n (n \geq n_0 \implies n^2/2 \leq c n)$.
        Porem, como $n_0 \geq 0$, escolhendo $n$ tal que $n \geq n_0$ e $n > 2c$, temos que $n \geq n_0$ e $n^2/2 \geq cn$, o que contradiz a nossa suposicao.
        
    \end{enumerate}
    
    \item
    Prove ou de um contra-exemplo para as afirmacoes abaixo:
    \begin{enumerate}
        \item
        $\lg{\sqrt{n}} = \mathcal{O}(\lg n)$
        
        \textit{\textbf{Resposta:}}\\
        Como $\lg{\sqrt{n}} = \lg{n^{1/2}} = \dfrac{\lg{n}}{2}$, eh facil ver que $\lg{\sqrt{n}} = \mathcal{O}(\lg n)$, e tambem $\lg n =  \mathcal{O}(\lg{\sqrt{n}})$.
        
        \item \label{item2}
        Se $f(n) = \mathcal{O}(g(n))$ e $g(n) = \mathcal{O}(h(n))$, entao $f(n) = \mathcal{O}(h(n))$
         
        \textit{\textbf{Resposta:}}\\
        Se $f(n) = \mathcal{O}(g(n))$ e $g(n) = \mathcal{O}(h(n))$, entao existem constantes positivas $c_1, c_2, n_1, n_2$ tais que 
        $$
        \forall n (n \geq n_1 \implies f(n) \leq c_1 g(n))
        $$
        e
        $$
        \forall n (n \geq n_2 \implies g(n) \leq c_2 h(n))
        $$
        Como $c_1, c_2 \geq 0$, entao tambem vale
        $$
        \forall n (n \geq n_1 \implies f(n) \leq c_1 g(n))
        $$
        e
        $$
        \forall n (n \geq n_2 \implies c_1 g(n) \leq c_1 c_2 h(n))
        $$
        Entao, escolhendo $n_0 = \max\{n_1, n_2\}$, e $c = c_1 c_2$, temos:
        $$
        \forall n (n \geq n_0 \implies f(n) \leq c h(n))
        $$
        Isto eh, $f(n) = \mathcal{O}(h(n))$\\
        
        \item
        Se $f(n) = \mathcal{O}(g(n))$ e $g(n) = \Theta(h(n))$, entao $f(n) = \Theta(h(n))$
         
        \textit{\textbf{Resposta:}}\\
        Dadas duas funcoes $t(n)$ e $s(n)$, temos que
        $t(n) = \Theta(s(n)) \iff t(n) = \mathcal{O}(s(n)) \wedge s(n) = \mathcal{O}(t(n))$.\\
        Seja $f(n) = n$, $g(n) = n^2$ e $h(n) = 2 n^2$. Eh claro que $g(n) = \Theta(h(n))$. Para mostrar que $f(n) = \mathcal{O}(g(n))$, basta escolher $c = 1$ e $n_0 = 1$, e entao
        $\forall n (n \geq n_0) \implies n \leq c n^2$.\\
        Ja sabemos por \ref{item2} que $f(n) = \mathcal{O}(h(n))$. Entao, para que $f(n) = \Theta(h(n))$, eh necessario que $h(n) = \mathcal{O}(f(n))$\\
        Vamos supor que seja o caso que $h(n) = \mathcal{O}(f(n))$. Entao existem constantes positivas $c$ e $n_0$ tais que
        $\forall n (n \geq n_0 \implies 2 n^2 \leq c n)$. Porem, como $n_0 \geq 0$, escolhendo $n$ tal que $n \geq n_0$ e $n > \dfrac{c}{2}$, temos que $2n^2 > cn$, contradizendo nossa hipotese.\\
        
        \item
        Suponha que $\lg(g(n)) > 0$ e que $f(n) \geq 1$ para todo $n$ suficientemente grande.
        Neste caso, se $f(n) = \mathcal{O}(g(n))$ entao $\lg(f(n)) = \mathcal{O}(\lg(g(n)))$
         
        \textit{\textbf{Resposta:}}\\
        A funcao $\lg n$ eh estritamente crescente, pois a sua primeira derivada eh $\dfrac{\lg e}{n}$, que eh positiva em todo o dominio.\\
        Como $f(n) \geq 1$ para todo $n$ suficientemente grande, entao existe uma constante $k \in \mathbb{R}$ tal que $n \geq k \implies f(n) \geq 1 \implies \lg (f(n)) \geq 0$.\\
        Se $f(n) = \mathcal{O}(g(n))$, isso significa que existem constantes positivas $c_1$ e $n_1$ tais que
        $$
        \forall n (n \geq n_1 \implies f(n) \leq c_1 g(n))
        $$
        Entao, sendo $n_0 = \max\{n_1, k\}$, vale que
        $$
        \forall n (n \geq n_0 \implies \lg(f(n)) \leq \lg (c_1 g(n)))
        $$
        Dado que $\lg (c_1 g(n)) = \lg (c_1) + \lg (g(n))$, entao, escolhendo $c \in R^+$ tal que
        $c \geq \dfrac{\lg (c_1)}{\lg (g(n))} + 1 \implies
        c \lg(g(n)) \geq \lg (c_1) + \lg (g(n))$, temos que:
        $$
        \forall n (n \geq n_0 \implies \lg (f(n)) \leq c \lg (g(n)))
        $$
        Isto eh, $\lg (f(n)) = \mathcal{O}(\lg (g(n)))$\\
        
        
    \end{enumerate}

\end{enumerate}

\end{document}