\documentclass{article}

% Language setting
% Replace `english' with e.g. `spanish' to change the document language
\usepackage[brazil]{babel}

% Set page size and margins
% Replace `letterpaper' with `a4paper' for UK/EU standard size
\usepackage[letterpaper,top=2cm,bottom=2cm,left=3cm,right=3cm,marginparwidth=1.75cm]{geometry}

% Useful packages
\usepackage{mathtools}
\usepackage{amsmath}
\usepackage{amsfonts}
\usepackage{graphicx}
\usepackage{algpseudocode}
\usepackage{algorithm}
\usepackage[colorlinks=true, allcolors=blue]{hyperref}
\usepackage{enumitem}
\usepackage{bm}
\usepackage{setspace}
\doublespacing
\title{
    Lista 04 - MAC0320\\
    Introdução à Teoria dos Grafos
}
\author{
    João Pedro Lukasavicus Silva\\
    \href{mailto:joao.lukasavicus.silva@usp.br}
    {joao.lukasavicus.silva@usp.br}\\
    Nº USP: 9276940\\
    Bacharelado em Ciências da Computação
}

\begin{document}
\maketitle

\begin{enumerate}[itemsep=3ex, label=E\textbf{\arabic*}.]
    \setcounter{enumi}{15}
    \item 
    Provar (nos moldes da prova vista em aula para o algoritmo de Kruskal) que o algoritmo descrito a seguir constrói uma árvore geradora de custo mínimo.

    \underline{ALGORITMO CORTA GASTOS}

    Entrada: Grafo conexo $G = (V, A)$, com custos $c(a) \geq 0$ em cada aresta $a \in A$.\\
    Saída: Árvore ótima $T$ (árvore geradora de custo mínimo).

    \begin{enumerate}[label=\arabic*.]
        \item (Ordenação) Ordene as arestas de $G$ em ordem não-crescente de seus custos.\\
        Chame-as de $a_1, a_2, \ldots, a_m$, sendo $c(a_1) \geq c(a_2) \geq \ldots \geq c(a_m)$.

        \item $T \leftarrow G$.

        \item Para $i = 1$ até $m$ faça\\
        se $T - a_i$ é conexo, então $T \leftarrow T - a_i$.

        \item Devolva $T$.
    \end{enumerate}

    \textbf{Resposta:} Seja $T$ um subgrafo de $G$ construído pelo algoritmo descrito. Claramente $T$ é um subgrafo conexo minimal, com $V(T) = V(G)$, e portanto $T$ é uma árvore geradora de $G$. Vamos provar que $T$ é uma árvore ótima.

    Suponha que $A(G) \setminus A(T) = \{e_1, e_2, \ldots, e_k\}$, e que $e_i$ foi escolhido (para não compor a $T$) antes de $e_j$ se $i < j$.\\
    Escolha uma árvore ótima $T^*$ tal que $|(A(T) \cap A(T^*))|$ seja máximo.\\
    Note que:
    \begin{align*}
        A(T^*) \cap (A(G) \setminus A(T)) &= (A(T^*) \cap A(G)) \setminus A(T)\\
        A(T^*) \cap (A(G) \setminus A(T)) &= A(T^*)  \setminus A(T)\\
    \end{align*}
    Suponha que $A(T^*)  \setminus A(T) = \emptyset$. Então $A(T^*) \subset A(T)$. Porém, como tanto $T$ quanto $T^*$ são ambos grafos conexos minimais, isso implica em $A(T) = A(T^*)$. Então temos que $c(T) = c(T^*)$, e como $T^*$ é uma árvore ótima, então $T$ também será.

    Suponha então que $A(T^*)  \setminus A(T) \neq \emptyset$. Note que $A(T^*)  \setminus A(T) = A(T^*) \cap (A(G) \setminus A(T))$.\\
    O conjunto $A(T^*) \cap (A(G) \setminus A(T))$ é o conjunto das arestas de $G$ que foram selecionadas pelo algoritmo para não fazerem parte de $T$, mas que estão em $T^*$. Estamos sob a suposição de que este conjunto não é vazio.\\
    Seja então $e_j$ a primeira aresta a ser selecionada para ser "excluída" de $G$ (para formar $T$) tal que $e_j$ esteja em $T^*$ (isso significa que as arestas $e_1, \ldots, e_{j-1}$ não pertencem a $T^*$), e sejam $\{u, v\}$ os vértices nos quais $e_j$ incide. Seja $P$ o único caminho em $T$ de $u$ a $v$. Note que o caminho $P$ tem pelo menos uma aresta, digamos $a$, que não pertence a $T^*$ (pois caso contrário $T^*$ teria um circuito). Como a aresta $e_j$ foi escolhida para ficar fora de $T$, e $a$ não foi ($a \in A(T)$), então $c(e_j) \geq c(a)$. De fato, basta notar que $G - (\{e_1, \ldots, e_{j-1}\} \cup \{a\})$ é conexo, e portanto, se tivéssemos $c(e_j) < c(a)$, então a aresta $a$ teria sido escolhida pelo algoritmo.\\
    Seja $T^\prime \coloneqq T^* + a - e_j$. Como $T^*$ é uma árvore, e portanto um grafo acíclico maximal, então $T^* + a$ contém exatamente um circuito, e vimos que esse circuito contém $e_j$. Então $T^* + a - e_j$ é um grafo conexo acíclico cujos vértices são exatamente os vértices de $G$, e portanto é uma árvore geradora de $G$.\\
    Como $T^*$ é uma árvore ótima, então $c(T^\prime) = c(T^*)$, e portanto $T^\prime$ também é uma árvore ótima. Mas $T^\prime$ contém mais arestas em comum com $T$ do que $T^*$ (isto é, $|A(T^\prime) \cap A(T)| > |A(T^*) \cap A(T)|$), uma contradição à escolha de $T^*$. Logo, devemos ter $A(T^*)  \setminus A(T) \neq \emptyset$, e já vimos que isso implica em $T = T^*$, e então $T$ será uma árvore geradora ótima.
\end{enumerate}

\end{document}